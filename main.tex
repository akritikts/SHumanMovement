%%
%% This is file `sample-acmtog.tex',
%% generated with the docstrip utility.
%%
%% The original source files were:
%%
%% samples.dtx  (with options: `acmtog')
%% 
%% IMPORTANT NOTICE:
%% 
%% For the copyright see the source file.
%% 
%% Any modified versions of this file must be renamed
%% with new filenames distinct from sample-acmtog.tex.
%% 
%% For distribution of the original source see the terms
%% for copying and modification in the file samples.dtx.
%% 
%% This generated file may be distributed as long as the
%% original source files, as listed above, are part of the
%% same distribution. (The sources need not necessarily be
%% in the same archive or directory.)
%%
%%
%% Commands for TeXCount
%TC:macro \cite [option:text,text]
%TC:macro \citep [option:text,text]
%TC:macro \citet [option:text,text]
%TC:envir table 0 1
%TC:envir table* 0 1
%TC:envir tabular [ignore] word
%TC:envir displaymath 0 word
%TC:envir math 0 word
%TC:envir comment 0 0
%%
%%
%% The first command in your LaTeX source must be the \documentclass command.
\documentclass[acmtog]{acmart}

%%
%% \BibTeX command to typeset BibTeX logo in the docs
\AtBeginDocument{%
  \providecommand\BibTeX{{%
    \normalfont B\kern-0.5em{\scshape i\kern-0.25em b}\kern-0.8em\TeX}}}

%% Rights management information.  This information is sent to you
%% when you complete the rights form.  These commands have SAMPLE
%% values in them; it is your responsibility as an author to replace
%% the commands and values with those provided to you when you
%% complete the rights form.
\setcopyright{acmcopyright}
\copyrightyear{2018}
\acmYear{2018}
\acmDOI{XXXXXXX.XXXXXXX}


%%
%% These commands are for a JOURNAL article.
\acmJournal{TOG}
\acmVolume{37}
\acmNumber{4}
\acmArticle{111}
\acmMonth{8}

%%
%% Submission ID.
%% Use this when submitting an article to a sponsored event. You'll
%% receive a unique submission ID from the organizers
%% of the event, and this ID should be used as the parameter to this command.
%%\acmSubmissionID{123-A56-BU3}

%%
%% The majority of ACM publications use numbered citations and
%% references.  The command \citestyle{authoryear} switches to the
%% "author year" style.
%%
%% If you are preparing content for an event
%% sponsored by ACM SIGGRAPH, you must use the "author year" style of
%% citations and references.
\citestyle{acmauthoryear}

%%
%% end of the preamble, start of the body of the document source.
\begin{document}

%%
%% The "title" command has an optional parameter,
%% allowing the author to define a "short title" to be used in page headers.
\title{Spatiotemporal human movement in indoor workspaces: A privacy preserving analysis}

%%
%% The "author" command and its associated commands are used to define
%% the authors and their affiliations.
%% Of note is the shared affiliation of the first two authors, and the
%% "authornote" and "authornotemark" commands
%% used to denote shared contribution to the research.
\author{Ben Trovato}
\authornote{Both authors contributed equally to this research.}
\email{trovato@corporation.com}
\orcid{1234-5678-9012}
\author{G.K.M. Tobin}
\authornotemark[1]
\email{webmaster@marysville-ohio.com}
\affiliation{%
  \institution{Institute for Clarity in Documentation}
  \streetaddress{P.O. Box 1212}
  \city{Dublin}
  \state{Ohio}
  \country{USA}
  \postcode{43017-6221}
}

\author{Lars Th{\o}rv{\"a}ld}
\affiliation{%
  \institution{The Th{\o}rv{\"a}ld Group}
  \streetaddress{1 Th{\o}rv{\"a}ld Circle}
  \city{Hekla}
  \country{Iceland}}
\email{larst@affiliation.org}

\author{Valerie B\'eranger}
\affiliation{%
  \institution{Inria Paris-Rocquencourt}
  \city{Rocquencourt}
  \country{France}
}

\author{Aparna Patel}
\affiliation{%
 \institution{Rajiv Gandhi University}
 \streetaddress{Rono-Hills}
 \city{Doimukh}
 \state{Arunachal Pradesh}
 \country{India}}

\author{Huifen Chan}
\affiliation{%
  \institution{Tsinghua University}
  \streetaddress{30 Shuangqing Rd}
  \city{Haidian Qu}
  \state{Beijing Shi}
  \country{China}}

\author{Charles Palmer}
\affiliation{%
  \institution{Palmer Research Laboratories}
  \streetaddress{8600 Datapoint Drive}
  \city{San Antonio}
  \state{Texas}
  \country{USA}
  \postcode{78229}}
\email{cpalmer@prl.com}

\author{John Smith}
\affiliation{%
  \institution{The Th{\o}rv{\"a}ld Group}
  \streetaddress{1 Th{\o}rv{\"a}ld Circle}
  \city{Hekla}
  \country{Iceland}}
\email{jsmith@affiliation.org}

\author{Julius P. Kumquat}
\affiliation{%
  \institution{The Kumquat Consortium}
  \city{New York}
  \country{USA}}
\email{jpkumquat@consortium.net}

%%
%% By default, the full list of authors will be used in the page
%% headers. Often, this list is too long, and will overlap
%% other information printed in the page headers. This command allows
%% the author to define a more concise list
%% of authors' names for this purpose.
\renewcommand{\shortauthors}{Trovato and Tobin, et al.}

%%
%% The abstract is a short summary of the work to be presented in the
%% article.
\begin{abstract}
  A clear and well-documented \LaTeX\ document is presented as an
  article formatted for publication by ACM in a conference proceedings
  or journal publication. Based on the ``acmart'' document class, this
  article presents and explains many of the common variations, as well
  as many of the formatting elements an author may use in the
  preparation of the documentation of their work.
\end{abstract}

%%
%% The code below is generated by the tool at http://dl.acm.org/ccs.cfm.
%% Please copy and paste the code instead of the example below.
%%
\begin{CCSXML}
<ccs2012>
 <concept>
  <concept_id>10010520.10010553.10010562</concept_id>
  <concept_desc>Computer systems organization~Embedded systems</concept_desc>
  <concept_significance>500</concept_significance>
 </concept>
 <concept>
  <concept_id>10010520.10010575.10010755</concept_id>
  <concept_desc>Computer systems organization~Redundancy</concept_desc>
  <concept_significance>300</concept_significance>
 </concept>
 <concept>
  <concept_id>10010520.10010553.10010554</concept_id>
  <concept_desc>Computer systems organization~Robotics</concept_desc>
  <concept_significance>100</concept_significance>
 </concept>
 <concept>
  <concept_id>10003033.10003083.10003095</concept_id>
  <concept_desc>Networks~Network reliability</concept_desc>
  <concept_significance>100</concept_significance>
 </concept>
</ccs2012>
\end{CCSXML}

\ccsdesc[500]{Computer systems organization~Embedded systems}
\ccsdesc[300]{Computer systems organization~Redundancy}
\ccsdesc{Computer systems organization~Robotics}
\ccsdesc[100]{Networks~Network reliability}

%%
%% Keywords. The author(s) should pick words that accurately describe
%% the work being presented. Separate the keywords with commas.
\keywords{datasets, neural networks, gaze detection, text tagging}


%%
%% This command processes the author and affiliation and title
%% information and builds the first part of the formatted document.
\maketitle

\section{Introduction}
Reopening workplace spaces will inevitably increase contact, interaction, group gathering, and close contacts making it harder to maintain social distancing, moreover, rely on social distance for safety. This, in turn, may raise the risk of transmission inside the workplace. Changes in office layout and increased air ventilation can be useful, but the increase in exposure due to increased person-to-person contact cannot be ignored. According to the World Health Organisation (WHO), the virus that is responsible for human-human transmission may stay viable after aerosolisation in the indoor environment. As a result, while returning to the office spaces, the impacts of human mobility inside the workspace and its impact on the workplace's ability to pose minimal risk must be extensively explored.



This research seeks to model human mobility inside four distinct types of workspaces (a chemistry lab on a university campus, an IT office, a hospital ward, and a factory) while accounting for spatial density, occupancy schedules, and numerous indoor activities. Understanding the mobility of employees at a workplace will enable us to understand their behaviour and identify the day-to-day tasks they undertake as an element of work. Apart from conventional mitigation strategies such as physical distancing, improved ventilation, face masks, and personal hygiene, this will provide methods for establishing coping mechanisms to prevent exposure risk rather than treating such instances as outliers. Because this study aims to forecast human mobility inside a facility, it can be combined with various air quality index measurements to forecast transmission risk, providing numerical clarity on how crowded a workspace or point of interest within the workspace can be while still posing minimal transmission risk. Previous research in this area has focused on either understanding occupant behaviour or exploring the effect of human activities on airflow patterns in the instance of a single person model in a controlled setting. There is a lot of research on using Wi-Fi access point data to model human mobility inside buildings \cite{qian2016decimeter}, \cite{meneses2012large} but not much work has been done in a detailed manner for specific spaces, such as a single lab or a workspace. In terms of studying the influence of human movement on indoor airflow patterns, most experiments have been carried out in a plotted methodology where the trajectory of the person was already planned \cite{mahaki2021comparing}, \cite{wu2022transient}. But the challenge is to correlate the behaviour of occupants with the airflow patterns in the space while preserving the privacy of the occupants. We perform privacy-preserving analysis in the wild, where the trajectory of participants and the air-quality index are the only data points collected. The context or the labels (such as the schedule of meetings) of events occurring in the workspace (such as the roster of employees working from the office) during the monitoring period are not known. This introduces a lot of apprehension in the process of modelling and generalising movement behaviour, which is dealt with by devising a set of uncertainty-aware rules to interpret the trajectories. To the best of our knowledge, this is the first work that studies the movement behaviour of occupants in an indoor office space and interweaves it with the airflow patterns in the presence of airborne disease. We explore the various aspects of indoor human movement throughout time and space by answering the following questions:
\begin{itemize}
    \item Do employees have a habit of visiting similar locations in the workplace?
    \item Do employees use the same routes to get from one location to another within the office?
    \item Is there a difference in the pattern of human movement during the day or between days?
    \item Is there a relationship between human movement and the density of airborne pathogens in different areas indoors?
    \item Are there any hot pockets (preferred stations, printing desk) inside the office space?
    \item Do the hot pockets get congested/crowded at regular intervals?
    \item Do the common footpaths inside the workspace pose a higher risk of transmission at all times?
    \item Do factors like sitting location, interconnecting doors, crowdedness and air quality influence human movement patterns indoors (geometric and non-geometric factors)?
\end{itemize}



Preventing the risk of infection and close contact is now part of workplace safety, posing a new challenge for sustaining workplace quality, promoting a sense of fulfilment and satisfaction, and enhancing productivity. A safer workplace will not only encourage employees to return in comfort but will also boost their performance. Because our work considers multiple kinds of workplaces, developing context-based workplace safety guidelines will serve a broader community. Therefore, the main contributions of this work are:
\begin{itemize}
    \item Indoor human movement study, we present an analysis of the spatiotemporal trajectories of employees inside the workspace.
    \item Event-based human-environment interaction approximation, we perform an event-based correlation between the human mobility trends and the airflow patterns inside the workspace.
    \item Spatial Context-based safety guidelines, we demonstrate practical implications of the relations between human movement and airflow patterns across workspaces.
\end{itemize}
The rest of this paper is organised as follows, we begin by reviewing the recent research work in this domain and identify the gaps in the literature. Then we introduce the utilised terminologies and describe the collected dataset after which we explain the methodology, elaborate on the analysis, and the results and discuss the relations between human movement and airflow patterns across workspaces before summing it up in the conclusion.


%%%%%%%%%%%%%%%%%%%%%%%%%%%%%%%%%%%%%%%%%%%%%%%%%%%%%%%%%%%%%%%%%%%%%%%%%%%%%%%%%%


\section{Related work}
With the outbreak of the pandemic and the struggle to reinstate office spaces, a lot of research has been done to reduce the risk of exposure in various types of indoor environments. Although the majority of developments in this field can be classified into three categories: safer indoor navigation, contact tracing and avoidance, and building occupancy simulation, there has also been some research to investigate the effects of human movement on air particles in an indoor space.

\subsection{Safer indoor traversal}

Costa and Ge et al. introduced a graph-based indoor-outdoor path technique that examines both the exterior graph of buildings and streets and the interior graph of the entrances and exits within each building to construct a unified, context-aware path \cite{costa2019caprio}. Jensen and Nielsen et al. created an algorithm that can traverse both indoor and outdoor spaces and yield the true shortest path between two arbitrary points \cite{jensen2016outdoor}. Salgado \& Cheema et al. used the Global Category Nearest Neighbour approximation method to conduct category aware multi-criteria route planning, which provides an ideal route between two indoor places by taking into consideration the inside, the outdoors, and semi-indoors \cite{salgado2018efficient}.

\subsection{Contact tracing and contact avoidance}

Trivedi et al. developed WiFiTrace, a network-centric approach to contact tracing that relies on passive WiFi sensing with no client-side involvement. They used WiFi network logs collected by enterprise networks for performance and security monitoring to reconstruct device trajectories for contact tracing \cite{trivedi2021wifitrace}. YYue et al. propose a neural network model termed Spatio-Temporal Episodic Memory for COVID-19 (STEMCOVID) that uses contact tracking data to detect infectious asymptotic instances. The model encodes a collective Spatio-temporal episodic memory of individuals and contains an effective parallel search mechanism based on the fusion Adaptive Resonance Theory (ART) \cite{hu2020silent}. Avoiding contact may also be accomplished by minimising congestion, which can be achieved with the help of congestion forecasting, a field that has experienced tremendous growth during the pandemic. ABy utilising the building floor plan and modelling human trajectory as a Gaussian distribution, Anastasiou and Costa et al created a platform for indoor congestion generation to facilitate congestion forecasting in indoor settings \cite{anastasiou2021epicgen}. Using Wi-Fi access points, Abrishami and Kumar constructed a model for estimating foot traffic for retailers inside a shopping mall \cite{abrishami2018using}. Islam and Gandhi et al. offer a collaborative filtering strategy based on tensor factorization that uses granular real-time foot traffic data to determine foot traffic for a point of interest \cite{islam2021spatiotemporal}. Using game theory, Xie and Luan et al devised a crowdsourcing-based indoor navigation model that encourages users to use navigation services. \cite{xie2021game}.

\subsection{Building occupancy simulation}
Sydora et al. provide a systematic simulation-based technique for calculating the infection risk for inhabitants of a building under various scenarios of building utilisation, as well as a virus transmission model that evaluates the possible infection transmission risk based on occupant behaviour \cite{sydora2022building}. Ciunkiewicz et al. devised an agent-based simulation (ABS) framework for localised settings that is highly configurable. This ABS enables extensive control over COVID-19's fast-growing epidemiological features, as well as information on risk and the impacts of both pharmaceutical and non-pharmacological interventions \cite{ciunkiewicz2022agent}. The findings of simulations can help facility administrators make better decisions and can also be utilised as inputs for a decision support system. 

\subsection{Studying indoor occupant behaviour}
Lee et al. investigate the effects of spatial and temporal restrictions in a facility, which might limit occupants' capacity to physically remove themselves, by conducting a survey to see if and how such constraints affect occupants' physical distancing behaviours in K–12 educational facilities \cite{lee2022understanding}. Jayjarah et al. developed a methodology for predicting the Likelihood of Future Non-Conformance (LFNC), which is based on the hypothesis that the likelihood of future deviations in movement behaviour is positively correlated to the intensity of trajectory deviations observed in the user's recent past and that the likelihood of future deviations increases if the user's strong-ties have also recently exhibited such non-conformant movement behaviour using longitudinal indoor location data from a university campus \cite{jayarajah2018predicting}. Chen et al. present a large-scale systematic analysis in the context of urban revisitation and re-check-in, demonstrating people's periodic behaviours and regularities by leveraging a localisation dataset to model urban revisitation and identifying features concerning POI visitation patterns, POI background information, user visitation patterns, user preference, and users \cite{chen2020will}.

\subsection{Investigating the impact of human movement on air particles in indoor spaces}
Using optical particle counters (OPS) for particulate concentration monitoring and resuspending source strengths of particles with characteristics of human-building interactions, Um et al. investigated the effects of occupant behaviour on indoor particle concentrations in daycare centres in four daycare centres in South Korea \cite{um2022occupant}.  Wu et al. performed a series of experiments to sample the nanoparticle concentration in the breathing zone of a sitting thermal breathing manikin (STBM) in order to investigate transient variability in nanoparticle concentration. They discovered that continuous exposure to a sitting person can result in a 2.88 (1.24) percent increase in nanoparticle concentration \cite{wu2022transient}. Mahaki et al. used a human-sized plate, cylinder, and detailed manikin to study the effects of human movement on airflow patterns around local ventilation hoods both experimentally and numerically, with the movements consisting of back-and-forth movements near an exhaust hood using a 3-D sonic anemometer to measure air velocity in front of the hood opening. The results revealed that the turbulence created by the object's motions contained distinct air velocity peaks in the near field of the exhaust hood, both supporting and inhibiting the suction flow \cite{mahaki2021comparing}.



Although some research has been done on modelling human movement for indoor environments and analysing the impact of human movement on airflow patterns inside a space, it has all been conducted in a controlled environment and thus does not take into account the spatiotemporal elements that may limit the occupant's ability to maintain a safe distance at all times or influence occupant behaviour, which may, in turn, drive the airflow pattern. Most studies consider the overall picture of the building, but what happens inside during working hours or scheduled activities, and its impact on controlling or lowering the risk of infection, are yet unknown. There are broad rules for personal hygiene and social distance, but in a space that is used by a variety of individuals throughout the day, as well as visitors and passers-by, the assumption that "if everyone does their bit, we'll all be safe" may not be sufficient.
Therefore the following gaps exist in the literature:
\begin{itemize}
    \item Lack of understanding about the spatiotemporal human movement inside different kinds of workspaces.
    \item Lack of a framework that correlates the indoor airflow pattern of a workspace with their respective human movement trends.
    \item Lack of understanding of the practical implications of the relations between human movement and airflow patterns across workspaces.
\end{itemize}


%%%%%%%%%%%%%%%%%%%%%%%%%%%%%%%%%%%%%%%%%%%%%%%%%%%%%%%%%%%%%%%%%%%%%%%%%%%%%%%%%%%%%


\section{Methodology}

Modifying the template --- including but not limited to: adjusting
margins, typeface sizes, line spacing, paragraph and list definitions,
and the use of the \verb|\vspace| command to manually adjust the
vertical spacing between elements of your work --- is not allowed.

\subsection{Terminologies}
\subsubsection{Human Movement}
The term human movement in context of this study refers to the movement of employees across grids for the sake of this research, taking into account the presence of numerous persons in a single grid as well as one or more employees travelling across various grids at the same time. This research does not take into consideration an employee's mobility inside a grid.

\subsection{Data Acquisition and Processing}
Employees who worked from the office were asked to complete a 15-question survey twice a day, which included questions regarding their seating location, most-used walking path, amount of social interactions, and movement patterns while leaving the workstation. This was conducted over a four-week period, yielding a total of 173 replies, with roughly 40\% staff involvement and an average of seven daily responses.

 We conducted the study at four sites, including a chemistry wet lab at RMIT University, an IT office, a Covid-19 ward inside a hospital and a factory, the and the data collection process was approved by the Human Research Ethics Committee at our University. The duration of data collection was different for different sites and the data was collected in two forms, participant trajectories using BLE tags and a survey.
 
Participants: Participants were recruited for each site. Employees/students at each site were briefed about the study as well as the data collection process. Those who volunteered to participate in the study were asked to fill and sign consent forms which were collected and stored in a locked box to preserve the privacy of the participants. Details about the consent form..... After they returned the signed consent forms, they were asked to take a Bluetooth tag to carry with themselves throughout the study. The participants were asked to carry the same tag for the whole peried of the study. No participant information associated with the tag was obtained. The participants were also asked to complete a 15-question survey twice everyday. The participants had to use their email-address to register for the daily survey reminders but their email-address was not correlated with any other information and was solely used to send survey reminders. No personal information was recorded in the survey either.

Collected data: Movement data: To record the movement of participants in their workspaces, they were asked to carry Kontakt Asset tags in their pockets for the period of the study. Kontakt tags are Bluetooth based sensors that emit RSSI signals that are detected and collected by gateways. The gateways have a range of 50m and are able to extract the location of the tags with respect to the room/space. The RSSI signals are recorded every 5 seconds for each tags and were extracted usingn the Kontakt API. The gateways were placed as such that they do not cover the restrooms in the office. Using these RSSI signals a KNN model was trained in order to predict the location of a participant in the workspace. The trajectory of the participants were then predicted using a time step of 20s.
Survey data: The participants were asked to fill a 15-question survey twice everyday for the duration of the study. This survey inlcuded questions about their movement, interactions, comfort as well as whether or not they were wearing masks. 

\section{Analysis of workspace behaviour}



\section{Results and Discussion}



\section{Authors and Affiliations}



\section{Rights Information}



\section{CCS Concepts and User-Defined Keywords}

Two elements of the ``acmart'' document class provide powerful
taxonomic tools for you to help readers find your work in an online
search.

The ACM Computing Classification System ---
\url{https://www.acm.org/publications/class-2012} --- is a set of
classifiers and concepts that describe the computing
discipline. Authors can select entries from this classification
system, via \url{https://dl.acm.org/ccs/ccs.cfm}, and generate the
commands to be included in the \LaTeX\ source.

User-defined keywords are a comma-separated list of words and phrases
of the authors' choosing, providing a more flexible way of describing
the research being presented.

CCS concepts and user-defined keywords are required for for all
articles over two pages in length, and are optional for one- and
two-page articles (or abstracts).

\section{Sectioning Commands}

Your work should use standard \LaTeX\ sectioning commands:
\verb|section|, \verb|subsection|, \verb|subsubsection|, and
\verb|paragraph|. They should be numbered; do not remove the numbering
from the commands.

Simulating a sectioning command by setting the first word or words of
a paragraph in boldface or italicized text is {\bfseries not allowed.}

\section{Tables}




%%
%% The acknowledgments section is defined using the "acks" environment
%% (and NOT an unnumbered section). This ensures the proper
%% identification of the section in the article metadata, and the
%% consistent spelling of the heading.
\begin{acks}
To Robert, for the bagels and explaining CMYK and color spaces.
\end{acks}

%%
%% The next two lines define the bibliography style to be used, and
%% the bibliography file.
\bibliographystyle{ACM-Reference-Format}
\bibliography{base}

%%
%% If your work has an appendix, this is the place to put it.
\appendix



\end{document}
\endinput
%%
%% End of file `sample-acmtog.tex'.
